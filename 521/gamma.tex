\documentclass{protokoll}
\newcommand{\assistent}{}
\newcommand{\versuch}{$\gamma$-Spektroskopie mit Szintillations- und Halbleiterdetektoren}
\newcommand{\nummer}{P521}

\begin{document}

\section{Einleitung}
Das Ziel dieses Versuchs ist es, die $\gamma$-Spektroskopie mit Szintillations-
und Ge-Halbleiterdetektoren zu untersuchen. Die charakteristischen Eigenschaften
wie Energieaufl�sung und Nachweiswahrscheinlichkeit der beiden Detektortypen
werden bestimmt und mit einander verglichen. Als Anwendung des Ge-Detektors wird
abschie�end eine Bodenprobe auf Spuren von Radioaktivit�t untersucht.

\section{Theoretische Grundlagen}
\subsection{Radioaktivit�t}
% zerfall, quellen, nat�rliche, zerfallsreihen, gammastrahlunng

\subsection{Wechselwirkung von $\gamma$-Strahlung mit Materie}
% photoeffekt, compton, paarbild., abh�ngigkeit des wq dieser effekte von der energie
% nd von Z des absorbers
$\gamma$-Strahlung kann auf verschiedene Arten mit Materie wechselwirken:
\begin{description}
    \item[Photoeffekt] Das Photon wechselwirkt mit dem Atom (bzw. dem Gitter bei
        Festk�rpern) und gibt seine gesamte Energie an dieses ab. Dabei wird
        ein Elektron frei, dass dann die kinetische Energie $W = h\nu - W_B$ hat,
        wobei $W_B$ die Bindungsenergie des Elektrons ist.

        F�r den Wirkungsquerschnitt dieser Wechselwirkung gilt:
        \begin{align}
            \sigma_{ph} \propto Z^5 E_\gamma^{-\frac{7}{2}}
            \label{eqn:wq-photo}
        \end{align}
    \item[Compton-Effekt] Der Compton-Effekt tritt auf, wenn das Photon mit
        einem freien oder quasi-freien (schwach gebundenen) Elektron
        wechselwirkt. Das Photon st��t elastisch (in dem Sinne, das keine
        Energie umgewandelt wird) auf das Elektron, Energie und Impuls der
        beiden Teilchen ergeben sich aus der relativistischen
        Energie-Impuls-Beziehung. Die Energie vor und nach dem Sto� h�ngt wie
        folgt zusammen:
        \begin{align}
            \frac{E_{nach}}{E_{vor}} = \frac{1}{1 + \frac{h\nu}{m_ec^2}(1-\cos
            \vartheta)}
            \label{eqn:compton-energie}
        \end{align}
        wobei $\vartheta$ der Streuwinkel ist.

        Da die Energie von $\gamma$-Quanten deutlich gr��er als die
        Bindungsenergie der Elektronen ist, k�nnen letztere bei derartiger
        Streuung als quasi-frei angesehen werden. F�r den Wirkungsquerschnitt
        gilt (wenn wir annehmen, dass $E_\gamma \gg m_ec^2$):
        \begin{align}
            \sigma_c \propto \frac{Z}{E_\gamma}
            \label{eqn:wq-compton}
        \end{align}

        Der Grenzfall geringer Photonenenergien nennt sich Thomson-Streuung.
        Dabei wird, wie man Formel \ref{eqn:compton-energie} leicht entnehmen
        kann, die Photonenenergie nicht ver�ndert, der Wirkungsquerschnitt
        betr�gt:
        \begin{align}
            \sigma_{th} = Z\frac{2\pi}{3}r_0^3
            \label{eqn:wq-thomson}
        \end{align}
        wobei $r_0$ der klassische Elektronenradius ist.
\end{description}

\subsection{Szintillationsdetektor}

\subsection{Halbleiterdetektor}
\subsubsection{B�ndermodell}
%blabla

\subsection{Impulsh�henspektrum}

\subsection{Vielkanalanalysator}

\subsection{Detektorcharakteristiken}

\subsection{Vergleich der beiden Detektortypen}

\subsection{Anwendungen der Detektoren}

\subsection{Termschemata und $\gamma$-Energien der verwendeten Quellen}

\section{Versuchsaufbau, -durchf�hrung und Beobachtungen}

\section{Zusammenfassung}

\begin{appendix}
\Literatur{quellen}
\end{appendix}

\end{document}
