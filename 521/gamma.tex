\documentclass{protokoll}
\newcommand{\assistent}{}
\newcommand{\versuch}{$\gamma$-Spektroskopie mit Szintillations- und Halbleiterdetektoren}
\newcommand{\nummer}{P521}

\begin{document}

\section{Einleitung}
Das Ziel dieses Versuchs ist es, die $\gamma$-Spektroskopie mit Szintillations-
und Ge-Halbleiterdetektoren zu untersuchen. Die charakteristischen Eigenschaften
wie Energieaufl�sung und Nachweiswahrscheinlichkeit der beiden Detektortypen
werden bestimmt und mit einander verglichen. Als Anwendung des Ge-Detektors wird
abschie�end eine Bodenprobe auf Spuren von Radioaktivit�t untersucht.

\section{Theoretische Grundlagen}
\subsection{Radioaktivit�t}
% quellen, nat�rliche, zerfallsreihen

Die Begriffe Radioaktivit�t, radioaktiver Zerfall und Kernzerfall beschreiben die Eigenschaft instabiler Atomkerne sich spontan unter Energieabgabe umzuwandeln. Die freiwerdende Energie wird als ionisierende Strahlung (energiereiche Teilchen und/oder Gammastrahlung) abgegeben. Dem entsprechend unterscheidet man drei verschiedene Arten von radioaktivem Zerfall.

\subsubsection*{$\alpha$-Zerfall}
Ein Atomkern der Massenzahl $A$ und Kernladungszahl $Z$ kann unter Aussendung eines $\alpha$-Teilchens, also eines Heliumkerns, zerfallen: 
\begin{align}
^A_ZX \to ^{A-4}_{Z-2}Y + ^{4}_{2}He^{++}
\end{align}
Der Zerfall ist energetisch m�glich, falls $m(A,Z) > m(A-4, Z-2) + m(\alpha)$ gilt.

Ein $\alpha$-Teilchen stellt einen sehr stabilen Zustand zweier Neutronen und zweier Protonen dar. Finden sich also die vier Konstituenten in einem Atomkern zusammen, so haben sie eine hohe Bindungsenergie, m�ssen aber noch das attraktive Kernpotential �berwinden.
Auf Grund des Tunneleffektes gibt es jedoch eine endliche Aufenthaltswahrscheinlichkeit des $\alpha$-Teilchens au�erhalb der Potentialbarriere. Hier �berwiegt das absto�ende Coulombpotential.
Die Energie der $\alpha$-Teilchen $E_\alpha$ ist, da es sich hier um einen Zweik�rper-Zerfall handelt, direkt von der Energie des Kerns abh�ngig, also diskret. Es ist die M�glichkeit der Kernanregung zu beachten, $E_\alpha$ kann also varrieren. 

\subsubsection*{$\beta$-Zerfall}
Der $\beta$-Zerfall subsummiert drei Zerf�lle, bei denen sich $Z$ um eins �ndert, $A$ jedoch konstant bleibt. Beim $\beta^-$-Zerfall zerf�llt ein Neutron innerhalb des Kerns in ein Proton, ein Elektron sowie ein Anti-Elektronneutrino:
\begin{align}
n \to p + e^- + \bar\nu_e
\end{align}
Dieser Zerfall ist bei $m(A,Z) > m(A,Z+1)$ energetisch m�glich.

Beim $\beta^+$-Zerfall hingegen zerf�llt ein gebundenes Proton in ein Neutron, ein Positron und ein Elektronneutrino:
\begin{align}
p \to n + e^+ + \nu_e
\end{align}
Dieser Zerfall ist bei $m(A,Z) > m(A,Z-1) + 2 m_e$ energetisch m�glich.

Des Weiteren existiert noch die M�glichkeit des Elektroneneinfangs. Da die Elektronen im Atom eine endliche Aufenthaltswahrscheinlichkeit im Kern haben, k�nnen diese Elektronen von einem Proton eingefangen werden. Hierbei entsteht ein Neutron und ein Elektronneutrino:
\begin{align}
p + e^- \to n + \nu_e
\end{align}
Dieser Zerfall ist bei $m(A,Z) > m(A,Z-1) + E^{e^-}_{Bind}$ energetisch m�glich, wobei $E^{e^-}_{Bind}$ die Bindungsenergie des Elektrons darstellt. In der jeweiligen Elektronenschale wird eine L�cke hinterlassen. Der Atomkern ist nun angeregt.

\subsubsection*{$\gamma$-Zerfall}
Zu guter letzt besteht die M�glichkeit, dass ein angeregter Atomkern in einen energetisch tiefer gelegenen Zustand �bergeht und dabei ein Photon der Differenzenergie aussendet analog zu den Vorg�ngen in der Atomh�lle. Die bei Kern�berg�ngen erreichbaren Energien der $\gamma$-Quanten $E_{\gamma}$ liegen jedoch wesentlich h�her (keV- oder MeV-Bereich).

% noch weiter


\subsection{Wechselwirkung von $\gamma$-Strahlung mit Materie}
% photoeffekt, compton, paarbild., abh�ngigkeit des wq dieser effekte von der energie
% nd von Z des absorbers
$\gamma$-Strahlung kann auf verschiedene Arten mit Materie wechselwirken:
\begin{description}
    \item[Photoeffekt] Das Photon wechselwirkt mit dem Atom (bzw. dem Gitter bei
        Festk�rpern) und gibt seine gesamte Energie an dieses ab. Dabei wird
        ein Elektron frei, dass dann die kinetische Energie $W = h\nu - W_B$ hat,
        wobei $W_B$ die Bindungsenergie des Elektrons ist.

        F�r den Wirkungsquerschnitt dieser Wechselwirkung gilt:
        \begin{align}
            \sigma_{ph} \propto Z^5 E_\gamma^{-\frac{7}{2}}
            \label{eqn:wq-photo}
        \end{align}
    \item[Compton-Effekt] Der Compton-Effekt tritt auf, wenn das Photon mit
        einem freien oder quasi-freien (schwach gebundenen) Elektron
        wechselwirkt. Das Photon st��t elastisch (in dem Sinne, das keine
        Energie umgewandelt wird) auf das Elektron, Energie und Impuls der
        beiden Teilchen ergeben sich aus der relativistischen
        Energie-Impuls-Beziehung. Die Energie vor und nach dem Sto� h�ngt wie
        folgt zusammen:
        \begin{align}
            \frac{E_{nach}}{E_{vor}} = \frac{1}{1 + \frac{h\nu}{m_ec^2}(1-\cos
            \vartheta)}
            \label{eqn:compton-energie}
        \end{align}
        wobei $\vartheta$ der Streuwinkel ist.

        Da die Energie von $\gamma$-Quanten deutlich gr��er als die
        Bindungsenergie der Elektronen ist, k�nnen letztere bei derartiger
        Streuung als quasi-frei angesehen werden. F�r den Wirkungsquerschnitt
        gilt (wenn wir annehmen, dass $E_\gamma \gg m_ec^2$):
        \begin{align}
            \sigma_c \propto \frac{Z}{E_\gamma}
            \label{eqn:wq-compton}
        \end{align}

        Der Grenzfall geringer Photonenenergien nennt sich Thomson-Streuung.
        Dabei wird, wie man Formel \ref{eqn:compton-energie} leicht entnehmen
        kann, die Photonenenergie nicht ver�ndert, der Wirkungsquerschnitt
        betr�gt:
        \begin{align}
            \sigma_{th} = Z\frac{2\pi}{3}r_0^3
            \label{eqn:wq-thomson}
        \end{align}
        wobei $r_0$ der klassische Elektronenradius ist.
\end{description}

\subsection{Szintillationsdetektor}

\subsection{Halbleiterdetektor}
\subsubsection{B�ndermodell}
%blabla

\subsection{Impulsh�henspektrum}

\subsection{Vielkanalanalysator}

\subsection{Detektorcharakteristiken}

\subsection{Vergleich der beiden Detektortypen}

\subsection{Anwendungen der Detektoren}

\subsection{Termschemata und $\gamma$-Energien der verwendeten Quellen}

\section{Versuchsaufbau, -durchf�hrung und Beobachtungen}

\section{Zusammenfassung}

\begin{appendix}
\Literatur{quellen}
\end{appendix}

\end{document}
